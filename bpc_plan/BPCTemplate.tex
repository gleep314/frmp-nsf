% NSF proposal generation template style file.
% based on latex stylefiles written by Stefan Llewellyn Smith and
% Sarah Gille, with contributions from other collaborators.
%
% Details BPC provided by Colleen Lewis.
\documentclass{proposalnsf}

\usepackage[dvipsnames]{xcolor}
\usepackage{enumitem,amssymb}
\newlist{rubricitem}{itemize}{2}
\setlist[rubricitem]{label=$\square$}

% See this file for a set of pre-defined journal abbreviations
%\input{journal-abbreviations.tex} 

\newcommand{\degrees}{$\!\!$\char23$\!$}
\renewcommand{\refname}{\centerline{References cited}}

% This handles hanging indents for publications
\def\rrr#1\\{\par
\medskip\hbox{\vbox{\parindent=2em\hsize=6.12in
\hangindent=4em\hangafter=1#1}}}

\def\baselinestretch{1}

\begin{document}

\color{Turquoise}
Blue text shows the instructions that should be replaced or deleted when writing your plan.\\

\textbf{FAQ:}
\begin{itemize}
    \item The template can be used once per activity.
    \item Template use is not required. It is designed to help PIs provide the required information.
    \item Activities of each PI can be overlapping or disjoint. If PI activities are overlapping, you could have a shared Context and Goals section. 
\end{itemize}

\begin{center}
\color{black}
{\Large{\bf Individual BPC Plan}}\\*[3mm]
\color{Turquoise}
{\bf Proposal Title} \\*[3mm]

PI Names \\
More PI Names

\end{center}

\color{Turquoise}
\noindent
\textbf{Getting Started?} PIs without BPC experience could plan to (1) participate in BPC education, (2) participate in existing departmental BPC activities and/or (3) create a Departmental BPC Plan if one does not exist. The BPC activities do not need to relate to the content of the grant. \\

PIs can replicate the Example BPC Activities (Separate doc) by including the relevant text from the example in their BPC Plan. Each section below has possible subheadings to enable reviewers to identify the relevant content in the Individual BPC Plan. 

\section*{Activity [N]}

\color{Black}
\section{Context \& Goals}
\textbf{Context:} \\ 
\textbf{Goal:} \\
\textbf{Activity Motivation:} \\

\color{Turquoise}
When you complete this section, you should have met these expectations from BPCnet.org: 
\begin{rubricitem}
    \item At least one specific, measurable, attainable, relevant, and time-bound (SMART) goal is provided for the plan. 
    \item For each activity, there is an explanation for why it is likely to be effective for BPC. 
    \item The expected benefit addresses a problem identified by currently available local and/or national data (which should be provided in the context section).
\end{rubricitem}

\color{Black}
\section{Intended Population}
\textbf{Activity Participants:} \\
\textbf{Participant Recruitment:} \\

\color{Turquoise}
When you complete this section, you should have met these expectations from BPCnet.org: 
\begin{rubricitem}
    \item For each activity, the intended participants are identified (e.g., demographic and age/level such as high school, graduate students). PIs might be the participants if the activity focuses on learning about BPC.
    \item For each activity, when relevant, the procedure for recruiting participants is provided and is feasible. For example, it may be feasible because (1) the recruiting strategy is detailed OR (2) explicit commitments are provided from relevant partners OR (3) evidence is provided of past successful recruiting efforts.
\end{rubricitem}

\color{Black}
\section{Strategy}
\textbf{Responsibilities of PIs:} \\
\textbf{Activity Content:} \\
\textbf{Activity Budget:} \\

\color{Turquoise}
When you complete this section, you should have met these expectations from BPCnet.org: 
\begin{rubricitem}
    \item For each activity, the role of each PI is clearly described. 
    \item For each activity, there is a clear description and a timeline. \item For each activity, relevant funding sources are identified as needed. Note: PIs can request funding for their Individual BPC Plan activities. The costs for the BPC activities are separate from the stated budget limits for proposals. PIs will submit for review a budget for the cost of any BPC activities with justification at award time. 
    \item Activity descriptions avoid the pitfall of assuming that underrepresented groups are homogeneous or deficient in some way. For example, an activity description might say that “women are more likely than men to want to contribute to society” (Lewis, et al., 2019)  and wouldn’t say that “all women want to contribute to society.” Additionally, activity descriptions should not suggest that BPC requires lowering standards.  
\end{rubricitem}

\color{Black}
\section{Preparation}

\color{Turquoise}
When you complete this section, you should have met this expectation from BPCnet.org: 
\begin{rubricitem}
    \item The PI responsibilities are realistic because of the preparation of the PIs. This preparation might include identifying (1) relevant prior experience and any lessons learned, (2) training plans, (3) a BPC expert who will partner with or coach the PIs, or (4) a resource that articulates and guides the necessary steps.
\end{rubricitem}


\color{black}
\section{Evaluation}

\color{Turquoise}
When you complete this section, you should have met these expectations from BPCnet.org: 
\begin{rubricitem}
    \item For each activity, feedback and data will be collected and used to improve the activities. 
    \item For each activity, the impact of activities will be measured.
    \item For each activity, evaluation data will be shared in the annual report. 
\end{rubricitem}
    

% --------------- WORKS CITED (10pt FONT) ---------------------

%\footnotesize
%\begin{thebibliography}{aa}

%\bibitem{amau} Maurer, Andrew B. \emph{\LaTeX \ Template for the National Science Foundation's Graduate Research fellowship Program}.

%\end{thebibliography}
\end{document}

% -------------------------------------------------------------

% -------------------------------------------------------------
