\documentclass[11pt]{article}

%!TEX root = ../project_description/project_description.tex
\usepackage[USenglish]{babel}

\usepackage[T1]{fontenc}      
\usepackage[utf8]{inputenc}

\usepackage{amsmath,amsthm}
\usepackage{amsfonts}
\usepackage{amssymb}
\usepackage{stmaryrd}
\usepackage{graphicx}
% \usepackage{titlesec}
\usepackage[margin=1in]{geometry}
\usepackage{subcaption}

\usepackage{xspace}
\newcommand{\ie}{i.e.\@\xspace}
\newcommand{\eg}{e.g.\@\xspace}

\makeatletter
\newcommand*{\etc}{%
    \@ifnextchar{.}%
        {etc}%
        {etc.\@\xspace}%
}
\makeatother

\makeatletter
\newcommand*{\etal}{%
    \@ifnextchar{.}%
        {et al}%
        {et al.\@\xspace}%
}
\makeatother

% \titleformat{\subsubsection}[runin]{\normalfont\large\bfseries}{\thesubsubsection}{1em}{}[]

\usepackage[textsize=scriptsize]{todonotes}

% \newlength{\arrow}
% \settowidth{\arrow}{\scriptsize$1000$}
\newcommand*{\goesto}[1]{\xrightarrow{#1}}
\newcommand*{\Xbar}{\overline{X}}
\newcommand*{\xbar}{\overline{x}}

\usepackage{hyperref}
\usepackage{listliketab}
\usepackage{array}
\usepackage{longtable}
\usepackage{enumitem}
\usepackage{sectsty}
\usepackage{url}
\usepackage{bibentry}
\usepackage{fancyhdr}
\usepackage{xparse}

\newenvironment{thoughts}{\begin{itemize}[itemsep=0pt]}{\end{itemize}}
\newcommand{\athought}[1]{\item \emph{#1}}

\RequirePackage[final]{microtype}
\DisableLigatures{encoding = T1, family = tt* }

\bibliographystyle{plain}


\begin{document}

    \setcounter{page}{1}
    \begin{center}
        {\Large {\bf Broadening Participation in Computing Plan}}
    \end{center}

	This is a collaborative proposal between three institutions, Iowa State University, Drake University, and Grinnell College.  These institutions are located in Central Iowa, all within a 50 mile radius of each other.  Drake University and Grinnell College are primarily undergraduate institutions (PUI) and Iowa State University is a Research 1 university.  This proposal leverages existing experience and expertise in the three institutions to enhance the participation in computing at all three institutions, especially in underrepresented groups.

    There are two main components of this plan.  First, the 
    
    %%%%%%%%%%%%%%%%%%%%%%%%%%%%%%%%%%%
    \section{Activity: Data Collection}
    %%%%%%%%%%%%%%%%%%%%%%%%%%%%%%%%%%%
    At all three institutions data collection of relevant data for broader participation in computing is either inaccurate, difficult to access, or non existent.
    In this activity, the PIs at each of the three institutions will examine existing data collection methods and any existing data.
    These methods and data are then analyzed for weaknesses and strengths.  Results of the analysis are shared between the three PIs for further review and reflection, culminating in a report and presentation that may be used by the {\it Department Regional Workshop} activity.
    Performing these actions will help each department focus its efforts on generating accurate data, but also inform each department where to focus recruitment energies.
    Department and institutional data is important, and once reliable data is available and underrepresented groups are identified, determining where to recruit potential students is also important.
    
    \subsection{Context and Goals}
    According to the Iowa Data Center, the state of Iowa consists of a population of 3,155,070, of whom 90.2\% are White, 6.1\% are Hispanic or Latino 3.6\% are Black or African American, and 2.5\% are Asian.
    The demographics of Iowa State University, Drake University, and Grinnell College are more diverse than the surrounding area, but there is a lack of data readily available for department demographics and retention.

    The goal of this activity is to collect and analyze the demographic data at each institution and identify any diversity blind spots at each institution.

    \subsection{Intended Population}
    \begin{itemize}
    	\item 
	    \textbf{Activity Participants}: N/A

	    \item 
	    \textbf{Participant Recruitment}: N/A
    \end{itemize}

    \subsection{Strategy}

    \begin{itemize}
    	\item 
	    \textbf{Responsibility of PIs}: Each PI will work with their respective institutions and departments to collect demographic and retention data.

	    \item 
	    \textbf{Activity Content}: At the end of the first year of data collection, the PIs will meet to evaluate and discuss the needs of each institution.
    \end{itemize}

    \subsection{Preparation}
    Each PI will work with their institution and department to develop a plan to collect and retain relevant data from their institution.
    The plan must ensure the data is kept up to date and is easy to access and analyze.

    \subsection{Evaluation}
    After the PIs have developed their data collection plans, they will schedule an initial meeting to discuss approaches and ensure the robustness of each plan.
    After each year of collecting data, the PIs will meet to ensure the data is being collected as planned and discuss possible next steps to broaden participation in computing at each institution.


    %%%%%%%%%%%%%%%%%%%%%%%%%%%%%%%%%%%%%%%%%%%%%%%%%%%
    \section{Activity: Central Iowa Diversity Workshop}
    %%%%%%%%%%%%%%%%%%%%%%%%%%%%%%%%%%%%%%%%%%%%%%%%%%%
    \subsection{Context and Goals}
    Broadening participation in computing is not only important at the PIs' institutions, but at every institution.
    The primary goal of this activity is to organize and host a diversity workshop that includes the computing departments of other academic institutions in the Central Iowa region.

    \subsection{Intended Population}
    \begin{itemize}
    	\item 
    	\textbf{Activity Participants}: Faculty, staff, and students from computing departments at higher education institutions in Central Iowa.
    	
    	\item 
    	\textbf{Participant Recruitment}: Each PI will use their connections with other institutions to probe interest in participating in these workshops.
    \end{itemize}

    \subsection{Strategy}

    \begin{itemize}
    	\item 
    	\textbf{Responsibility of PIs}: The PIs will work together to develop, organize, and market the workshops to surrounding institutions.
    	
    	\item 
    	\textbf{Activity Content}:
    	The workshop be at least a half-day and include several speakers and breakout discussion groups addressing the needs of broadening participation of computing in Central Iowa.
        During the workshop, there will be opportunities for each institution to reflect on their own needs for growth in diversity and draft an initial plan to address those needs.

    \end{itemize}

    \subsection{Preparation}
    During the first two years, PIs Klinge, Lathrop, and Osera will recruit keynote speakers in the Central Iowa region to participate in the workshop.
    During the third year, the PIs will meet regularly to organize and plan the workshop, which will be held at the end of the third year.

    \subsection{Evaluation}
    After the workshop, the PIs will evaluate its impact by examining the number of institutions attending and the number of participants.
    They will also distribute a questionnaire to the participants inquiring what aspects of the workshop were the most engaging and helpful.

\end{document}

