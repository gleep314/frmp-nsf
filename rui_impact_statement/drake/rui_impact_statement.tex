\documentclass[11pt]{article}

%!TEX root = ../project_description/project_description.tex
\usepackage[USenglish]{babel}

\usepackage[T1]{fontenc}      
\usepackage[utf8]{inputenc}

\usepackage{amsmath,amsthm}
\usepackage{amsfonts}
\usepackage{amssymb}
\usepackage{stmaryrd}
\usepackage{graphicx}
% \usepackage{titlesec}
\usepackage[margin=1in]{geometry}
\usepackage{subcaption}

\usepackage{xspace}
\newcommand{\ie}{i.e.\@\xspace}
\newcommand{\eg}{e.g.\@\xspace}

\makeatletter
\newcommand*{\etc}{%
    \@ifnextchar{.}%
        {etc}%
        {etc.\@\xspace}%
}
\makeatother

\makeatletter
\newcommand*{\etal}{%
    \@ifnextchar{.}%
        {et al}%
        {et al.\@\xspace}%
}
\makeatother

% \titleformat{\subsubsection}[runin]{\normalfont\large\bfseries}{\thesubsubsection}{1em}{}[]

\usepackage[textsize=scriptsize]{todonotes}

% \newlength{\arrow}
% \settowidth{\arrow}{\scriptsize$1000$}
\newcommand*{\goesto}[1]{\xrightarrow{#1}}
\newcommand*{\Xbar}{\overline{X}}
\newcommand*{\xbar}{\overline{x}}

\usepackage{hyperref}
\usepackage{listliketab}
\usepackage{array}
\usepackage{longtable}
\usepackage{enumitem}
\usepackage{sectsty}
\usepackage{url}
\usepackage{bibentry}
\usepackage{fancyhdr}
\usepackage{xparse}

\newenvironment{thoughts}{\begin{itemize}[itemsep=0pt]}{\end{itemize}}
\newcommand{\athought}[1]{\item \emph{#1}}

\RequirePackage[final]{microtype}
\DisableLigatures{encoding = T1, family = tt* }

\bibliographystyle{plain}


\begin{document}

    \setcounter{page}{1}
    \begin{center}
        {\Large {\bf RUI Impact Statement}}
    \end{center}

    \begin{center}
        {\Large Drake University}
    \end{center}

    %%%%%%%%%%%%%%%%%%%%%%%%%%%%%%%%%%
    \section{Drake University Profile}
    %%%%%%%%%%%%%%%%%%%%%%%%%%%%%%%%%%
    Located in Des Moines, IA, Drake University enrolls more than 3000 undergraduate students and more than 1800 masters students
    % \todo{is masters students the correct term?}
    and is divided into six colleges and schools.
    Drake University was founded in 1881 and is especially well-known for its Law School and Pharmacy and Health Sciences programs.
    The institution's largest school is the College of Arts and Sciences, consisting of over 47\% of student enrollment distributed over 42 majors.
    Drake attracts students from all over the world, with over 30 U.S.\ states and 28 countries represented.
    The College of Arts and Sciences maintains an 8:1 student-faculty ratio, and over 70\% of classes have fewer than 20 students enrolled.
    The faculty at Drake are passionate, and the small class sizes enhance pedagogy and active learning.
    As a result, Drake is ranked by U.S.\ News \& World Report in the top twelve institutions for undergraduate teaching in the Midwest.
    Drake students are ambitious; roughly one in every two students in the College of Arts and Sciences pursue a double major, triple major, or two or more minors.

    \textbf{Computer Science.}
    The computer science program at Drake University is still in its infancy but is rapidly developing; the number of majors has doubled every three years for over a decade.
    In 2018, there were 136 computer science majors at Drake in contrast to a single major in 2006.
    The program is housed in the Department of Mathematics and Computer Science along with five other programs including data analytics, which is a variation of the computer science major emphasizing data science.
    Combined, computer science and data analytics now have over 200 majors.
    Drake University is generously supportive of the rapidly growing programs.
    The number of tenure-track faculty at Drake has doubled
    % \todo{Can I say this since Chris is joining CS?}
    in the last four years and a search is currently underway for two more computer science faculty to start fall 2020.
    Moreover, the department recently moved into the newly constructed Collier-Scripps Hall which features cutting-edge technology and generous classroom space to support high-quality pedagogy.
    The computer science faculty are passionate about undergraduate education and have recently received multiple awards from the university for outstanding teaching effectiveness.

    \textbf{Research Environment.}
    Drake University provides generous support for faculty and student research.
    For example, the Drake Undergraduate Science Collaborative Institute (DUSCI) offers many programs to broaden participation in undergraduate research.
    DUSCI offers \$3000 fellowships to students to participate in full-time eight-week summer research projects under the supervision of Drake faculty members.
    For sixteen years, the university has also hosted the Drake University Conference on Undergraduate Research in the Sciences (DUCURS) which is a full-day conference in the Spring semester.
    In 2019, DUCURS hosted 8 oral presentations and 84 posters, highlighting the results of Drake undergraduate research projects during the previous year.
    The Drake Science Colloquium Series is another program hosted by DUSCI and features presentations from Drake faculty and other scholars on their current research.

    The Mathematics and Computer Science Department is also exploring methods to broaden and diversify participation in undergraduate research in computer science.
    Last year,
    % \todo{When did this start?}
    the department created various peer-mentored research projects that are overseen by faculty.
    This fall, there are 5 active projects involving more than 25 students, many of which are women and students from underrepresented groups.
    % \todo{Is this a true statement?}
    The department has also regularly funded student research through internal and external grants such as the Iowa Space Consortium.

    %%%%%%%%%%%%%%%%%%%%%%%%%%%%%%%%%%%%%%%%%%%%%%%
    \section{Impact of Project}
    %%%%%%%%%%%%%%%%%%%%%%%%%%%%%%%%%%%%%%%%%%%%%%%
    Drake University is the PI's first tenure-track position, and this project will lay the foundation of his undergraduate research laboratory.
    The PI plans to maintain a group of 3--6 active undergraduate research students in perpetuity, and this project provides the initial support needed to launch his research group.
    The PI also plans to support additional students through his negotiated startup funds and other competitive grants such as DUSCI fellowships.

    \textbf{Preparing Students for Advanced Degrees in STEM.}
    Although the department's student-led research projects have successfully broadened and diversified the number of students interested in research, the projects themselves often lack the depth and the faculty-mentoring necessary to adequately prepare students for advanced degrees in STEM.
    % \todo{Is this a true statement?}
    A major goal of this project is to create a cohort of undergraduate students, train them in cutting-edge molecular programming concepts, and mentor them through the scientific research process.
    During the project, three students per summer at Drake will be exposed to 9 weeks of full-time research and will continue part-time during the academic year.
    All of the students will be exposed to graduate-level research and actively collaborate with students from Iowa State University and Grinnell College, including a graduate student funded by the project at Iowa State University.
    Each of the students will be individually mentored by the PI and will be provided the depth of research experience necessary to prepare them for advanced degrees in STEM.
    % \todo{Consider talking about them being mentored by the other PI's}

    The PI is well-equipped to do this mentoring and already has a history of preparing students for graduate school.
    While he was a Visiting Assistant Professor at Carleton College, the PI mentored two students on an independent research project in collaboration with two graduate students from Iowa State University supervised by PI James Lathrop.
    All four students presented a poster of their work at the \emph{International Conference on DNA Computing and Molecular Programming} in August 2019 in Seattle, WA.
    They also have written a preliminary paper which will be submitted for publication early next year.
    One of these students is from an underrepresented group, and both are applying for graduate schools in computer science this year.

    \textbf{Broadening Participation in STEM Research.}
    Molecular programming is a magnet for talented students.
    For example, in September, 2019, the PI gave a guest lecture in the Junior/Senior Chemistry Seminar at Drake University titled \emph{Molecular Programming: an algorithmic approach to DNA nanotechnology} and was immediately approached by multiple students interested in undergraduate research opportunities.
    Similarly, when the PI sought applicants for four undergraduate research assistant positions for a summer project at Grinnell College in 2017, a diverse pool of 24 qualified students applied.

    The interdisciplinary nature of this project makes it attractive to a diverse population of students.
    During the first year of the project, the PI will teach a special topics course titled \emph{Molecular Programming and Nanoscale Self-Assembly} at Drake University.
    % \todo{Is this OK?}
    This course will expose a broad number of students to hands-on research as well as diversify the pool of qualified applicants for the project.

    The PI also has experience attracting students to STEM fields through his \emph{Introduction to Computer Science} courses at Carleton College and Grinnell College.
    While teaching these courses, multiple students from underrepresented groups approached the PI about their plans to add a computer science major after taking his class.
    Two of them said they were inspired by the deep connections computer science has to other disciplines such as chemistry and molecular biology as demonstrated by the PI's research.

    \textbf{Impact of Project on PI's Career}
    The project will have a strong impact on the PI's career.
    The project will enable the PI to travel to two conferences per year, which is essential during the first few years of an academic career.
    Furthermore, the project provides two months of research support to the PI per year to maintain his research productivity while also teaching a 3-3 course load.

    The PI's research agenda concerns bridging the gap of the theory and practice of molecular programming.
    This project is a natural next step to his research, and due to the interdisciplinary nature of the project, the collaborations with PI James Lathrop and PI Peter-Michael Osera are essential.
    Furthermore, the undergraduate students involved in the project will provide the support necessary to complete the research.

    \section{Plan for Finding Qualified Undergraduate Participants}
    In order to attract a diverse pool of talented students, the PI will teach a special topics course at Drake University on molecular programming during the 2020/2021 academic year.
    The course will help spark interest in the project as well as train students in the fundamentals of molecular programming before the first full-time summer of the project.
    To attract the interest of students outside the course, the PI will also advertise the project in fall 2020 when the department launches its student-led research projects.
    When appropriate, the PI will also personally encourage students from underrepresented groups to apply who may lack the self-confidence to consider joining the project.

    The criteria used for selecting undergraduate research assistants will include: academic success, passion for research, and clear interest in the project.
    First and second year students that demonstrate clear potential will also be considered.
    If applicants have similar qualifications, priority will be given to students from underrepresented groups.

    To assess the effect on the students participating on the project, the PI will monitor their academic achievements throughout the project and track how many participates go on to pursue a career in STEM or an advanced degree.

\end{document}