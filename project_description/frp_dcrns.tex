%!TEX root = project_description.tex

%%%%%%%%%%%%%%%%%%%%%%%%%%%%%%%%%%%%%%%%%
\section{FRP and Deterministic CRNs}
\label{sec:frp_dcrns}
%%%%%%%%%%%%%%%%%%%%%%%%%%%%%%%%%%%%%%%%%

Arguably the most well-known semantics for chemical reaction networks is \emph{deterministic}, where the molecular counts of each species is modeled as a continuously-varying concentration.
Moreover, these species concentrations evolve continuously according to a system of ordinary differential equations.
Deterministic CRNs are inherently \emph{analog} and are closely related to Shannon's general purpose analog computer (GPAC) model (CITE).
Traditionally deterministic CRNs and GPACs receive their input via an \

% 3. FRP and Deterministic CRNs
%     * Briefly introduce the input/output deterministic CRN model
%     * Draw connections between between the previous works on
%       chemical circuits and finite automata and FRP
%     * In particular, emphasize how these CRNs are literally signal
%       functions in the FRP sense that transform one type of signal
%       into another
%     * Drawing connections with how real biological systems use
%       molecular communication and react to input signals changing
%       such as the cell cycle switch
%     * Overview Ally's work on deterministic CRNs, the IOCRN type,
%       the Arrow implementations, and examples
%     * Show how the NAND gate and S-R latch can easily be implemented
%       in this characterization. Use the Yampa Arrow syntax to
%       demonstrate how the S-R latch can be implemented using
%       two NAND gates and the Arrow combinators
%     * Emphasize how the Bool type encapsulates the idea of a dual
%       rail encoding whereas the Double type is a continuous real
%       valued signal
%     * Discuss how STL (a continuous-space variant of LTL) is a natural
%       choice for a type system. Give examples of how the requirements
%       of the NAND gate and finite automata can be specified in STL
