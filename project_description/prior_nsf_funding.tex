%!TEX root = project_description.tex

%%%%%%%%%%%%%%%%%%%%%%%%%%%%%%%%%%%%%%%%
\section{Results from Prior NSF Support}
\label{sec:prior_nsf_support}
%%%%%%%%%%%%%%%%%%%%%%%%%%%%%%%%%%%%%%%%

{\bf T. Klinge} has no prior NSF support.
\vspace{1em}

\noindent
{\bf J. Lathrop} is a Co-PI on the current award CNS 1545028, CPS:Synergy: Safety-Aware Cyber-Molecular Systems, 10/15/2015 -- 8/31/2021, \$823,930.

\subsection*{Intellectual Merit}
The project combines methods from computer science and software engineering with methods from molecular biology to gain insights into how to develop cyber-molecular systems that are safe for use in dynamic and only partially understood environments.
The project has developed a refined formulation of what robustness means in deterministic chemical reaction networks. to  perform correctly even when crucial physical parameters are perturbed by small amounts and shown that nondeterministic finite automata can be simulated by deterministic chemical reaction networks that are provably robust in this sense.  
The project found that the use of advanced requirements engineering techniques for cyber-molecular systems enabled earlier removal of obstacles to safe operations.  Key results from the project also include the development of chemical reaction network models for standard safety building blocks needed for cyber-molecular systems \cite{jKlLaLu20,  cElLaLu17, cElKlLa17, jEKLLLM17}.
The project also implemented and evaluated in the laboratory a DNA origami-based reconfigurable nanosystem with potential as a force/energy diagnostic tool \cite{jMatHen16, oMath16}.    Publications describing these and additional results include \cite{cKlin16,  cHuaStu16, jCaLuSt18, cElKlLa17,  jHKLLL18}.
The project also supported graduate students to investigate the CRN-TAM model and developed a programming language and compiler for the model \cite{ALCH}.

\subsection*{Broader Impacts}
Eight Ph.D. students have been supported or partially supported by this award. Five of these have graduated since 2016:  Xiang Huang advised by J. Lutz (Ph.D. 2020, now an Assistant Professor at the University of Illinois at Springfield), Samuel Ellis co-advised by J. Lathrop and R. Lutz (Ph.D. 2017, now a Software Scientist at the NSF-funded Molecular Sciences Software Institute), Titus Klinge advised by J Lathrop and J Lutz  (Ph.D. 2016, now an Assistant Professor at Drake University, and prior appointments as Visiting Assistant Professor at Grinnell College and Carleton College), Adam Case advised by J. Lutz (Ph.D. 2016, now an Assistant Professor at Drake University), Don Stull advised J. Lutz (Ph.D. 2017, now a Lecturer at Iowa State University, and prior appointment at INRIA), Divita Mathur advised by E. Henderson (Molecular Biology Department) and J. Lutz (Ph.D. 2016, now at the U.S> Navel Research Laboratory).  Four Ph.D. students currently work in the Laboratory for Molecular Programming (LAMP).  

The project's REU supported research by two outstanding students, Gabrielle Ortman (graduated 2017)  and Chase Koehler (graduated 2018; co-authored a paper \cite{cKMHL18}).

J. Lutz developed  an undergraduate/graduate course on molecular programming and has taught it seven times with help from J. Lathrop, who taught it an eighth time in Fall, 2018.  As part of our educational outreach, J. Lathrop, J. Lutz and R. Lutz mentor students on molecular programming projects at two local undergraduate institutions, Grinnell College and Simpson College.
The investigators gave three tutorials on cyber-molecular systems at conferences (ASE 2015, ICSE 2016, RE 2018); a workshop at Simpson College and two invited keynotes (RE 2016, SAFECOMP 2018).
R. Lutz also gave the invited, public Dean's 2017 Spring Lecture at Iowa State University. \\

\noindent{\bf P.M. Osera} was the PI on award CCF 1651817, EAGER: Semi-Automated Type-Directed Programming, 2016--2019, \$159,991.
%
\subsection*{Intellectual Merit}
This project investigated the foundations of type-directed program synthesis along two dimensions: (1) extending the foundations to handle advanced typing features such as polymorphism and generalized algebraic datatypes and (2) realizing the foundations in a semi-automated program synthesis tool, \textsc{Scythe}, that supports type-directed programming in the Haskell programming language.

\subsection*{Broader Impacts}
This project supported 12 undergraduates for summer research opportunities over its lifetime.
Three of the alumni from this project are either in or applying to doctoral programs in computer science and/or mathematics.
The project also funded their travel to present their work at Midwest PL Summit as well as attend the Programming Languages Mentoring Workshop at PLDI and ICFP.