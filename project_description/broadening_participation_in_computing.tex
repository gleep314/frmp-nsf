%!TEX root = project_description.tex
\section{Broadening Participation in Computing}

% A strong BPC plan addresses at least the following five elements:
% (1) the context of the proposed activity,
% (2) intended target population(s),
% (3) plan of activities,
% (4) prior experience (if any) and/or intended preparation/training activities, and
% (5) plans for the measurement and dissemination of outcomes.
% All collaborators are expected to participate in BPC activities.
% More information, including metrics for BPC activities and examples, can be found at: 
% https://www.nsf.gov/cise/bpc/.
The PIs have substantial prior experience exposing students, researchers, and others to computing and computing with molecules using a variety of platforms.  Some examples of the PIs prior experience in generating interest in computing include:
\begin{enumerate}
    \item PI Klinge developed a molecular programming course at Grinnell College and Carleton College.
    \item PI Klinge supervised independent study in molecular programming for two students at Carleton College from underrepresented groups.
    \item PI Klinge and PI Lathrop twice conducted a molecular programming workshop to students and faculty at Simpson College.
    \item PI Lathrop has taught several classes in video game design and programming and integrating parts of the course with producing educational games for the Battleship Iowa museum. One student went on to create a game that is now permanently installed at the museum.  Another woman student in this class went on to create a game for kids 3 to 5 years old.  The game was presented at the Game Development Conference, and the student received a special award.
    \item PI Klinge has presented a tutorial on safe molecular programming at the Automated Software Engineering conference.
    \item PI Lathrop has mentored students at Simpson College participating in the Carver Bridge To STEM Success program.
  \item PI Osera has reserved 1--2 number of his summer research opportunities each year to qualified students from underrepresented groups and rising first- and second years.
  \item PI Osera has participated in the Grinnell Science Project over multiple years, a Presidential Award-winning program that encourages first-year students to pursue careers in STEM, in particular, those from underrepresented groups.
\end{enumerate}

Several activities are planned to help broaden participation in computing, and specifically molecular programming.
% Revolves around ICICL thing
\begin{enumerate}

    \item PI Klinge will develop and introduce an updated molecular programming course at Drake University.
    \item PI Klinge will teach a three-week outreach workshop to high school students on molecular programming at Carleton College.
    \item PI Klinge and PI Lathrop will travel to high schools in the central Iowa area to introduce students to molecular programming.  The PIs will especially try to focus on schools in rural areas where resources or expertise to teach or introduce molecular programming may not be as readily available as in college towns and larger cities.
    \item PI Lathrop teaches a game design and programming course once a year.  The PI will utilize the class to have students design and create games that demonstrate computing or molecular computing.
    \item PI Lathrop will travel to K1-12 schools to introduce students to computing and molecular computing using appropriate technology for the grade.  This may include using AR and VR technologies to help visualize computing paradigms, and/or games (possibly developed by students in a gaming course) to help illustrate technology and computing.
    \item PI Klinge will engage with the Boys and Girls club on the Drake campus to help increase participation in computing.
    \item PI Klinge will approach the Oakridge Neighborhood to engage low-income students with STEM and computing education through the STEM on the Ridge summer program.
  \item PI Osera will continue to participate in the Grinnell Science Project in the upcoming years to help attract underrepresented groups to STEM fields.
\end{enumerate}

Dissemination of research in this area can provide motivation for students to pursue advanced degrees in computer science and engineering.  PI Lathrop often uses information local talks at Iowa State University to attract students to computer science.  In addition, dissemination of this research at
\begin{enumerate}
    \item international conferences,
    \item local conferences,
    \item regional meetings and workshops,
    \item poster, conference publications, and journal publications, and
    \item open source software
\end{enumerate}
will give exposure to computing and computing technology.

