%!TEX root = project_description.tex

%%%%%%%%%%%%%%%%%%%%%%%%%%%%%%%%%%%%%%%%%%%%%%%%%%%%%%%%%%%%%%%%%%%%%%%%
\section{Tool Support and Software for Functional Reactive Molecular Programming in CRNs}
\label{sec:software_support}
%%%%%%%%%%%%%%%%%%%%%%%%%%%%%%%%%%%%%%%%%%%%%%%%%%%%%%%%%%%%%%%%%%%%%%%%

Several software artifacts will be produced to support the proposed research and ultimately make the results of the proposed research accessible to the scientific community.
During the first year of the proposed work, we will develop the following software artifacts.
\begin{enumerate}
	\item \textbf{Two Haskell Libraries}\\
    These libraries comprise the completed prototypes of FRP primitives for deterministic and stochastic CRN development.
	Our preliminary investigation and development of these libraries is discussed in Sections~\ref{sec:frp_dcrns} and~\ref{sec:frp_scrns}.
	We will use the data we collect from these prototypes to inform the development of the other software artifacts.
	
	\item \textbf{CRN Translation Tools}\\
    These tools will translate high-level CRNs developed in the aforementioned libraries into low-level internal CRN representations and other CRN file formats such as the Systems Biology Markup Language (SBML).
	This will ensure our Haskell libraries synergize with existing software tools such as Visual GEC and Matlab Simbiology.
\end{enumerate}

As discussed in Sections~\ref{sec:frp_dcrns} and~\ref{sec:frp_scrns}, the main software artifacts of our proposed research are two new languages that facilitate deterministic and stochastic CRN development in a functional reactive programming style.
These languages will include a type system inspired by temporal logics such as STL, LTL, and CSL for both deterministic and stochastic CRNs.
Along with these languages, we will develop appropriate support for these languages that build upon existing software such as Microsoft Visual Studio Code.
We envision creating:
\begin{enumerate}
	\item Basic editing support such as syntax highlighting and error reporting.
	\item Integrated simulation and testing tools for visualizing the I/O behavior of developed CRNs.
  \item Advanced editing support by leveraging PI Osera's prior work in type-directed program synthesis~\cite{osera:thesis:2015, osera:pldi:2015, frankle:popl:2015, osera:tyde:2019}.
\end{enumerate}

We anticipate other software tools will be needed as the research progresses, and these will also be released freely for public use.
The development of most of these software artifacts are ideal for engaging undergraduate researchers in the project.
