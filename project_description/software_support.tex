%!TEX root = project_description.tex

%%%%%%%%%%%%%%%%%%%%%%%%%%%%%%%%%%%%%%%%%%%%%%%%%%%%%%%%%%%%%%%%%%%%%%%%
\section{Tool Support and Software for Functional Reactive Molecular Programming in CRNs}
\label{sec:software_support}
%%%%%%%%%%%%%%%%%%%%%%%%%%%%%%%%%%%%%%%%%%%%%%%%%%%%%%%%%%%%%%%%%%%%%%%%

Several software artifacts will be produced to aid in the proposed research and make the research accessible to the scientific community.  Such tools will support a cohesive and reliable environment for research and development of the of the proposed research, and provide end-user tools to researchers and molecular programmers.  As discussed in Section~\ref{sec:broader_impacts}, these tools may be useful in a more general context such as analog computing using FRP.
Most of these tools will reside within an integrated development environment and the proposed research will make heavy use of an existing development environment that may be freely distributed to the research community.
The integrated development environment must include support for extending the environment to accommodate the proposed research.
Examples of such integrated environments include Visual Studio Code and Eclipse.

The proposed research includes creating the following software artifacts.

\begin{enumerate}
	\item FRMP library that comprises the of the system in Haskell for FRP
	
	\item <name?>  library for stochastic stuff both leader and FRP version
	
	\item Library for translating Haskell CRN representation to actual chemical reaction network.
	
	\item Plugin support for highlighting CRN specific types with appropriate error indication for types.  Tool tip indicates and suggestions for corrections.  (Could lead to synthetic programming?)
	
	\item plugin that supports export to common formats such as SBML so that CRN tools can be used to analyze resutlant CRNs.  Also allows programs such as VIsual GEC and Matlab simbiology to load produced CRNs.
	
	\item plugin to support simulation inside the IDE to allow immediate visual confirmation of the code.
	
	\item plugin that allows STL and LTL specifications to coesist with the code.  (What should this do?)
	
	\item It is anticipated that other plugin tools will be recognized as the research progresses, and such tools will also be developed and released.
	

\end{enumerate}
\todo{reword these items}

Many of these individual projects are ideal for undergraduate participation to help create the overall system.


% 5. Software Support for Functional Reactive Molecular Programming
%     * This may not need its own section
%     * Do not actually mention Cauldron
%     * Just mention that we propose to create a deliverable that
%       gives IDE tooling support for FRP in the MP context
