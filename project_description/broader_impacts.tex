%!TEX root = project_description.tex

% The Project Description also must contain, as a separate section within the narrative, 
% a section labeled **Broader Impacts**. 
% - should provide a discussion of the broader impacts of the proposed activities. 
% - Broader impacts may be accomplished through 
%     + the research itself, 
%     + through the activities that are directly related to specific research projects, or 
%     + through activities that are supported by, but are complementary to the project.

% NSF values the advancement of scientific knowledge and activities that contribute to the 
% achievement of societally relevant outcomes. Such outcomes include, but are not limited to:
% - full participation of women, persons with disabilities, and underrepresented minorities in 
%   science, technology, engineering, and mathematics (STEM);
% -  improved STEM education and educator development at any level; 
% -  increased public scientific literacy and public engagement with science and technology; 
% -  improved well-being of individuals in society; 
% -  development of a diverse, globally competitive STEM workforce; 
% -  increased partnerships between academia, industry, and others; 
% -  improved national security; 
% -  increased economic competitiveness of the US; and 
% -  enhanced infrastructure for research and education.

% The following will be taken into consideration when reviewing the broader impacts of the proposal:

% 1.  What is the potential for the proposed activity to:
%     a.  Advance knowledge and understanding within its own field or across different fields (Intellectual Merit); and
%     b.  Benefit society or advance desired societal outcomes (Broader Impacts)?

% 2.  To what extent do the proposed activities suggest and explore creative, original, or potentially transformative concepts?

% 3.  Is the plan for carrying out the proposed activities well-reasoned, well-organized, and based on a sound rationale? Does the plan incorporate a mechanism to assess success?

% 4.  How well qualified is the individual, team, or organization to conduct the proposed activities?

% 5.  Are there adequate resources available to the PI (either at the home organization or through collaborations) to carry out the proposed activities?

\section{Broader Impacts}

The objective of the proposed work in this proposal is to investigate and develop high-level techniques for molecular programming through functional reactive chemical networks.
The expected societal benefit of the project is to improve participation of students (including high school students) undergraduate students, graduate students, and researchers in computer science, mathematics, engineering, chemistry, and biology in molecular programming through higher-level and simpler molecular programming techniques.
The results of this research may also benefit developers and manufacturers of nano-devices, such as bio-compatible devices and drug delivery systems.
Research results will be published in journals and conferences, included in the investigators' undergraduate and graduate courses.
Software tools developed as part of the research will be made available through public software repositories.

%% Young peersons trained in research stuff
%% Undergrade research training
Due to the proposed collaboration between Iowa State University and two primarily undergraduate colleges located near Iowa State University, the proposal has an active educational research component for training undergraduate students in research in STEM areas. 
At each institution, the PI will engage undergraduate students in the proposed research, leading both to significant training for students in scientific research, and a deep understanding of molecular programming.
 

%Highschool involvement through stem meetings, classroom visits

%Public lectures tagetting young people, especially those not exposed to university reseach -- DesMoines examples

%Honors mentor programs


%% Research will allow more people to develop and research  molecular nano devices that use molecular programming
% Mostly due to the new techniques developed here

% Tools created for the research made pubically available.  (IDE?)

% potential to change the way researchers program analog devices  (understand and prescibe in a new way)

%  Mapping molecular programming languge to requirements (automatically?)

%  Accelerate pace of the field is moving by having new deveopment techniques and formal language to descibe these systems at a higher level.
