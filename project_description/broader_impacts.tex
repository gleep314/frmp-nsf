%!TEX root = project_description.tex

%%%%%%%%%%%%%%%%%%%%%%%%%%%%%%%%%
\section{Broader Impacts}
\label{sec:broader_impacts}
%%%%%%%%%%%%%%%%%%%%%%%%%%%%%%%%%

% The Project Description also must contain, as a separate section within the narrative, 
% a section labeled **Broader Impacts**. 
% - should provide a discussion of the broader impacts of the proposed activities. 
% - Broader impacts may be accomplished through 
%     + the research itself, 
%     + through the activities that are directly related to specific research projects, or 
%     + through activities that are supported by, but are complementary to the project.

% NSF values the advancement of scientific knowledge and activities that contribute to the 
% achievement of societally relevant outcomes. Such outcomes include, but are not limited to:
% - full participation of women, persons with disabilities, and underrepresented minorities in 
%   science, technology, engineering, and mathematics (STEM);
% -  improved STEM education and educator development at any level; 
% -  increased public scientific literacy and public engagement with science and technology; 
% -  improved well-being of individuals in society; 
% -  development of a diverse, globally competitive STEM workforce; 
% -  increased partnerships between academia, industry, and others; 
% -  improved national security; 
% -  increased economic competitiveness of the US; and 
% -  enhanced infrastructure for research and education.

% The following will be taken into consideration when reviewing the broader impacts of the proposal:

% 1.  What is the potential for the proposed activity to:
%     a.  Advance knowledge and understanding within its own field or across different fields (Intellectual Merit); and
%     b.  Benefit society or advance desired societal outcomes (Broader Impacts)?

% 2.  To what extent do the proposed activities suggest and explore creative, original, or potentially transformative concepts?

% 3.  Is the plan for carrying out the proposed activities well-reasoned, well-organized, and based on a sound rationale? Does the plan incorporate a mechanism to assess success?

% 4.  How well qualified is the individual, team, or organization to conduct the proposed activities?

% 5.  Are there adequate resources available to the PI (either at the home organization or through collaborations) to carry out the proposed activities?

Molecular programming is an interdisciplinary field aiming to algorithmically control the function and form of matter at the nanoscale.
Recent advancements in the area have led to new technologies such as biochemical nano-robots~\cite{jDoBaCh12} \todo{MOAR robot references!} and techniques to self-assemble arbitrary two- and three-dimensional nano-structures~\cite{jRoth06,jDDLHGS09,jDMTVCS09,jKOSY12,benson2015dna,Juneaav0655}.
Many molecular devices are now being developed using chemical reaction networks since they can automatically be compiled into DNA molecules to interface with biological systems~\cite{cSoSeWi09,jLYCP12,jCard13,jCDSPCS13,jSPSWS17,cBSJDTW17}.

The objective of the proposed work is to investigate and develop high-level techniques for molecular programming through functional reactive programming.
In addition, the proposed research will necessarily produce innovative tools for writing molecular programs, making it easier and more intuitive to create complex chemical reaction networks.
We expect the proposed research to benefit society by catalyzing the development of nanotechnologies with new, more natural techniques for molecular programming.
Moreover, since many applications of molecular programming are safety-critical~\cite{jEKLLLM17,cLutz18}, the proposed research promises to help ensure these technologies are correct, reliable, and safe by producing programming languages with type systems that are statically verified, ensuring that temporal properties are satisfied.
Since the approaches and techniques of the proposed research are broadly applicable to other fields, this investigation will also benefit research in synthetic biology, general purpose analog computers, and distributed models of computation.

All research in the proposed work will be made public, presented in appropriate conferences, and published in well-known journals.
Moreover, all software artifacts produced will be posted on public-facing website so that they may be used by the scientific community.


\subsection{Planned Broader Impact Activities}
% The proposed research may improve participation of students (including high school students) undergraduate students, graduate students, high school and college educators in a cross-disciplinary field.
The collaboration between faculty at a research 1 and two primarily undergraduate institutions provides a unique opportunity for collaboration on research and education at all levels, leveraging the diverse experiences of the investigators and collaborators.
The planned activities of this collaboration are outlined below.  

\textbf{Curriculum Development.}
All of the PIs have experience developing courses that give students research experience in the classroom.
PI Klinge developed and taught a molecular programming course at two different undergraduate institutions, Grinnell College and Carleton College, PI Lathrop developed and taught an undergraduate and graduate course in molecular programming at Iowa State University, and PI Osera developed and taught a special topics course in ... at Grinnell College.
\todo{PM, fix the previous sentence so that it is true}
The result of the proposed research will enhance these classes and their curriculum.
PI Klinge will also develop a new undergraduate research experience course at Drake University during the first year of the proposed work.
The proposed research also includes elements from programming languages curricula, which provides an opportunity to adapt existing programming classes at these institutions.

Beyond curriculum for college-level classes, the proposed research offers an opportunity at the high school level.
Material will be generated for workshops, educational talks, and classroom exercises to  motivate students in STEM areas.
PI Klinge and PI Lathrop have designed molecular workshops in the past for students at Simpson college.
These workshops were successful and stimulated participation in additional collaborations between Simpson college and Iowa State University.
By updating and creating new presentations and workshops, PI Klinge, PI Lathrop, and PI Osera will include programming language theory with examples in molecular programming.
These new presentations and workshops will be designed for multiple levels of expertise (including high school students) and target different disciplines in computer science and molecular programming.


\textbf{Undergraduate and Graduate Research Opportunities}
\begin{enumerate}
	\item Training undergraduate students for research at each institution with collaboration with other institutions
	\item Training graduate students at Iowa State University in research, including experience in teaching and leading undergraduate students in small research sub tasks under the supervision of the PIs
	\item Continued exposure of research through honors programs and other programs at the three institutions
	\todo{Add in other examples of programs or things we can do for research opportunities}
\end{enumerate}

\subsubsection*{Interdisciplinary Research and Education}
The PIs will explore using this technology to create interdisciplinary groups at the institutions to further advance molecular programming in other disciplines.  To this end, we will investigate if the proposed technologies and tools can further enhance involvement in multidisciplinary competitions such as the iGEM or BIOMOD competitions for students.

