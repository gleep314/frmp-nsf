%!TEX root = project_description.tex

%%%%%%%%%%%%%%%%%%%%%%%%%%%%%%%%%%%%%%%%%
\section{FRP and Stochastic CRNs}
\label{sec:frp_scrns}
%%%%%%%%%%%%%%%%%%%%%%%%%%%%%%%%%%%%%%%%%

When modeling molecular interactions in small volumes, the deterministic chemical reaction network is not often suitable for some situations.
Consider a small volume (such as mammalian cells) where the presence or absence of a single virus molecule may determine whether it lives or dies.
Chemical systems in these environments are inherently probabilistic, and \emph{stochastic mass action kinetics} is an appropriate~\cite{jGill77,oCSWB09} model instead.
In a stochastic CRN, molecular counts are modeled with a continuous-time Markov chain (CTMC), where each state of the Markov chain represents a vector of integral molecular counts, and the rates between each state are determined by the \emph{law of mass action}.
Stochastic CRNs are equivalent to many well-known models such as vector addition systems~\cite{oGins66,jKaMiWi67,jKarMil69,jNash73,jLero10,cLero12}, Petri nets~\cite{oPetr62,jMura89,oDavAll10,oReis13}, and population protocols~\cite{jAADFP06,jAAER07,jAnAsEi08,jAnAsEi08a,Doty2018}.

Molecular programmers commonly program stochastic CRNs in two different ways.
The first approach is to use a \emph{leader molecule} to help direct the computation.
Use of leaders are important in many distributed computing algorithms and have been thoroughly investigated in the context of population protocols~\cite{jAnAsEi08,Doty2018}.
In stochastic CRNs, a leader molecule is used to enable and disable target stages of the computation to guarantee modularity and prevent interference from different parts of the program.
Commonly, only one leader molecule exists in the solution at a time, and when one stage of the computation finishes, it decays into a new leader molecule that enables the next part of the algorithm.
This approach was used by~\cite{jSCWB08} to prove that stochastic CRNs are Turing complete with arbitrarily low probability of error.
Note that such a technique is impossible in a deterministic CRN because a species with strictly positive concentration will remain positive forever.
The second approach is to embrace the stochasticity of the model and harness it to perform computation.
For example, in~\cite{CAPPELLETTI202064}, the authors exploit the stochastic nature of these CRNs to implement arbitrary probability distributions at the nanoscale.
In~\cite{cWinfe19}, the author harnesses this stochasticity to solve a variety of optimization and constraint satisfaction problems.

Since programming with stochastic CRNs is fundamentally different than deterministic CRNs, the use of functional reactive programming will necessarily be different.
In the latter approach, stochastic CRNs are still intrinsically \emph{signal functions} with a continuous time domain and a discrete co-domain of integral molecular counts, and therefore the FRP style is still appropriate.
However, due to the stochasticity of these CRNs, the input and output signals must consist of probability distributions over possible state configurations.
This introduces another challenging layer of complexity to programming with stochastic CRNs.

We propose to develop a variant of \reactamole{} with signal function types such as a \hask{SCRN a b} for stochastic CRNs that natively supports types \hask{Bool} and \hask{Int} data types.
However, due to the extra layer of probability theory CRN development, our preliminary work for stochastic CRNs development explores the \emph{leader-directed} approach which is discussed in the following section.
We plan to use the data from this preliminary investigation to inform our development of the more sophisticated FRP approach over probabilistic signals.

Our initial work producing a variant of \reactamole{} for stochastic CRNs is based on the leader-directed approaches used in~\cite{jAnAsEi08,jSCWB08,Doty2018}.
We developed another Haskell library akin to \reactamole{} with a datatype \hask{SCRN a b} that represents a pure function, transforming inputs of type \hask{a} into outputs of type \hask{b}.
This library currently supports \hask{Int}s, dual-rail \hask{Bool}s, and pairs \hask{(a,b)}, and includes primitive stochastic CRNs inspired by~\cite{cChDoSo12} and various combinators such as \hask{>>>}, \hask{&&&}, and \hask{***}.
For example, the so-called ``hailstone'' function from the famous Collatz conjecture can be defined with:
\begin{haskellcode}
hailstone :: SCRN Int Int
hailstone = (is_even &&& (divide 2 &&& (multiply 3 >>> add 1))) >>> mux
\end{haskellcode}
Our prototype library enforces that each functions input/output types match.
For example, \hask{mux} has a type signature of \hask{SCRN (Bool (Int, Int)) Int} and connects the appropriate value to the output based on the value of the \hask{Bool} parameter.
Each stage of the stochastic CRN computation is implicitly directed by an unnamed leader molecule, generated when the \hask{SCRN a b} datatype is compiled to a low-level CRN.
Any stochastic CRN that computes a sufficiently complex function necessarily must have a positive probability of error.
Our library also ensures this probability of error is bounded using a clocking module similar to~\cite{jSCWB08}.

Although this preliminary work on leader-directed programming has been fruitful, it does not take full advantage of \emph{reactive} programming.
Ultimately, our goal is to extend this work to support full-fledge FRP support for probabilistic signals which is outlined in the following section.


%%%%%%%%%%%%%%%%%%%%%%%%%%%%%%%%%%%%%%%%%%
\subsection{FRP and Probabilistic Signals}
%%%%%%%%%%%%%%%%%%%%%%%%%%%%%%%%%%%%%%%%%%

To be written...

%     B. FRP and Probabalistic Signals
%         * Emphasize how both stochastic CRNs are inherently probabalistic
%           and how input signals are in reality, mapping time to a
%           distribution of possible CRN states
%         * Mention how the previously mentioned approach employs the leader
%           method to precisely control the ordering of events. In contrast,
%           biological systems oftem harness stochasticity to be a benefit
%         * We propose to investigate a more general approach to employ FRP
%           in this stochastic setting by regarding the signals of SCRNs to
%           be signals of "probability distributions"
%         * Since CSL is known to be a bisimulation for SCRNs, we will also
%           explore how a CSL type system may be employed to reason about
%           the complex abstractions of manipulating probability
%           distributions through FRP combinators