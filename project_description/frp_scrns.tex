%!TEX root = project_description.tex

%%%%%%%%%%%%%%%%%%%%%%%%%%%%%%%%%%%%%%%%%
\section{FRP and Stochastic CRNs}
\label{sec:frp_scrns}
%%%%%%%%%%%%%%%%%%%%%%%%%%%%%%%%%%%%%%%%%o

To be written...

% 4. FRP and Stochastic CRNs
%     * Briefly introduce the stochastic CRN model and constrast its
%       unique qualities with the deterministic I/O CRN model
%     * Discuss the multiple approaches that could be explored to
%       provide language support for SCRNs, in particular, the deterministic
%       function computation approach and the more general probability
%       distribution approach
%

\subsection{Leader-Directed Functional Programming}

To be written...

%     A. Leader-Directed Functional Programming
%         * Draw connections to current literature on deterministic function
%           computation, leader election, and Turing completeness results
%           that often use leader molecules to direct the computation
%         * Overview Bryce's work on stochastic CRNs, the SCRN type, the
%           Arrow implementations, etc.
%         * Show how the hailstone function can be easily implemented
%           using FRP combinators
%         * Explain how each stage of the computation may be compiled to
%           have an arbitrarily low chance of failure
%         * Contrast this approach with the IOCRN and how its inputs are
%           consumed during the computation
%         * Explain that leader-directed 
%

\subsection{FRP and Probabilistic Signals}

To be written...

%     B. FRP and Probabalistic Signals
%         * Emphasize how both stochastic CRNs are inherently probabalistic
%           and how input signals are in reality, mapping time to a
%           distribution of possible CRN states
%         * Mention how the previously mentioned approach employs the leader
%           method to precisely control the ordering of events. In contrast,
%           biological systems oftem harness stochasticity to be a benefit
%         * We propose to investigate a more general approach to employ FRP
%           in this stochastic setting by regarding the signals of SCRNs to
%           be signals of "probability distributions"
%         * Since CSL is known to be a bisimulation for SCRNs, we will also
%           explore how a CSL type system may be employed to reason about
%           the complex abstractions of manipulating probability
%           distributions through FRP combinators