%!TEX root = project_description.tex

% The following will be taken into consideration when reviewing the intellectual merit of the proposal:

% 1.  What is the potential for the proposed activity to:
%     a.  Advance knowledge and understanding within its own field or across different fields (Intellectual Merit); and
%     b.  Benefit society or advance desired societal outcomes (Broader Impacts)?

% 2.  To what extent do the proposed activities suggest and explore creative, original, or potentially transformative concepts?

% 3.  Is the plan for carrying out the proposed activities well-reasoned, well-organized, and based on a sound rationale? Does the plan incorporate a mechanism to assess success?

% 4.  How well qualified is the individual, team, or organization to conduct the proposed activities?

% 5.  Are there adequate resources available to the PI (either at the home organization or through collaborations) to carry out the proposed activities?


\section{Intellectual Merit}
Since molecular programming is significantly different from traditional programming approaches, this project will identify novel language constructs and methodologies that appropriately complement the unique advantages of molecular programming while abstracting away their error-prone low-level details.
Specifically, the proposed research will address the fragile and error-prone nature of molecular programming by utilizing techniques from functional reactive programming community.
The new techniques and methodologies in this proposed research will enhance the molecular program development process of other researchers, educators, and students and help them write programs that are correct and easy to maintain.  The proposed research is composed of three main thrusts:
\begin{enumerate}
	\item The syntax and semantics of a functional reactive programming language for chemical reaction networks.
	\item A LTL-based type system for the designed functional reactive programming language.
	\item Additional tool-based support to research programming CRNs using functional reactive programming techniques.
\end{enumerate}
