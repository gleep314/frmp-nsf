\documentclass[11pt]{article}

%!TEX root = ../project_description/project_description.tex
\usepackage[USenglish]{babel}

\usepackage[T1]{fontenc}      
\usepackage[utf8]{inputenc}

\usepackage{amsmath,amsthm}
\usepackage{amsfonts}
\usepackage{amssymb}
\usepackage{stmaryrd}
\usepackage{graphicx}
% \usepackage{titlesec}
\usepackage[margin=1in]{geometry}
\usepackage{subcaption}

\usepackage{xspace}
\newcommand{\ie}{i.e.\@\xspace}
\newcommand{\eg}{e.g.\@\xspace}

\makeatletter
\newcommand*{\etc}{%
    \@ifnextchar{.}%
        {etc}%
        {etc.\@\xspace}%
}
\makeatother

\makeatletter
\newcommand*{\etal}{%
    \@ifnextchar{.}%
        {et al}%
        {et al.\@\xspace}%
}
\makeatother

% \titleformat{\subsubsection}[runin]{\normalfont\large\bfseries}{\thesubsubsection}{1em}{}[]

\usepackage[textsize=scriptsize]{todonotes}

% \newlength{\arrow}
% \settowidth{\arrow}{\scriptsize$1000$}
\newcommand*{\goesto}[1]{\xrightarrow{#1}}
\newcommand*{\Xbar}{\overline{X}}
\newcommand*{\xbar}{\overline{x}}

\usepackage{hyperref}
\usepackage{listliketab}
\usepackage{array}
\usepackage{longtable}
\usepackage{enumitem}
\usepackage{sectsty}
\usepackage{url}
\usepackage{bibentry}
\usepackage{fancyhdr}
\usepackage{xparse}

\newenvironment{thoughts}{\begin{itemize}[itemsep=0pt]}{\end{itemize}}
\newcommand{\athought}[1]{\item \emph{#1}}

\RequirePackage[final]{microtype}
\DisableLigatures{encoding = T1, family = tt* }

\bibliographystyle{plain}


\begin{document}

    \setcounter{page}{1}
    \begin{center}
        {\Large {\bf Project Summary}}
    \end{center}
    \vspace*{1em}

    % \textbf{Overview.}
    % Molecular programming is an interdisciplinary field aiming to algorithmically control the function and form of matter at the nanoscale.
    % Recent advancements in the area have led to new technologies such as biochemical nano-robots and techniques to self-assemble arbitrary two- and three-dimensional nano-structures.
    % Many molecular devices are now being developed using \emph{chemical reaction networks} (\emph{CRNs}) since they can automatically be compiled into DNA molecules to interface with biological systems.
    % Although CRNs are Turing complete, they are akin to an unstructured assembly language, making even trivial computational tasks challenging to achieve.
    % This is due to the fact that the CRN model, designed over a half-century ago, was never intended to be used as a programming language.

    % This project will lay a foundation for the next generation of molecular programming by developing and studying a new high-level molecular programming language based on the functional reactive programming (\emph{FRP}) literature found in the programming languages community.
    % Chemical reaction networks are inherently \emph{reactive} and their behaviors can be naturally characterized as functional reactive programs.
    % Furthermore \emph{FRP} is a well-studied programming paradigm with associated theory and techniques that can easily be improved into the domain of \emph{CRNs}
    % This project will investigate (a) a functional reactive characterization of \emph{CRNs} with chemical signal generator primitives and high-order combinators that compile to traditional \emph{CRNs}, (b) a type system for this FRP language for CRNs inspired by linear temporal logic that enforces correctness guarantees of the network, and (c) additional tool-based support for programming \emph{CRNs} built on top of the functional reactive foundations we will establish.

    % \textbf{Intellectual Merit.}
    % % This project applies high-level techniques from another research discipline into molecular programming.
    % Since molecular programming is significantly different from traditional programming approaches, this project will identify novel language constructs and methodologies that appropriately complement the unique advantages of molecular programming while abstracting away their error-prone low-level details.
    % The new techniques and methodologies produced by this project will enhance the molecular program development process of other researchers, educators, and students and help them write programs that are correct and easy to maintain.
    % Moreover, the proposed work will advance the understanding of these programming language approaches by investigating alternative applications and uses of them in the field of molecular programming.
    % % Using typed chemical reaction networks and functional reactive programming techniques, the proposed work will extend and create new techniques beyond the current state of the art in the molecular programming paradigm.

    % \textbf{Broader Impacts.}
    % This project includes a substantial undergraduate research component at three institutions, two of which are primarily undergraduate institutions.
    % The PIs, graduate student, and undergraduate students will collaborate to develop the languages and any software necessary to perform the research.
    % This project will provide scientific research training and mentoring to all these students.
    % All techniques and software produced during the project will be published or made available online for public use.
    % The investigators and students will give tutorials and workshops at nearby colleges and reach out to high school students from underrepresented groups.
    % The PI's have experience reaching out to students and have recently co-taught molecular programming workshops and tutorials.
    % This project will also produce undergraduate curriculum incorporating the techniques and tools developed.

    % This project will also accelerate the development of nanotechnologies by producing next-generation language techniques for molecular programming.
    % Such advancements are likely to lead to smart therapeutics and other medical technologies that will broadly impact society as a whole.

\end{document}
