\documentclass[11pt]{article}

\newcommand{\thedoc}{Biographical Sketch}
\usepackage{doc}

\begin{document}

\startdoc{\shorttitle}

\section*{Professional Preparation}
\begin{tabular}[h]{cccc}
  University of Washington &
    \begin{minipage}{7cm}
      \begin{center}
        Computer Science \\ Applied and Computational Math Sciences \\ Comparative History of Ideas
      \end{center}
    \end{minipage}
    & \pbox{20cm}{B.S. \\ B.A.} & 2006 \\
  \\
  University of Pennsylvania & Computer Science & \pbox{20cm}{M.S. \\ Ph.D.} & 2015
\end{tabular}

\section*{Appointments}
\begin{itemize}
  \item \textbf{Grinnell College}, 2015--present \\
    \emph{Assistant Professor}, Computer Science Department
  \item \textbf{University of Pennsylvania}, 2008--2015 \\
    \emph{Graduate Research Assistant}, Department of Computer and Information Science
  \item \textbf{Microsoft Corporation}, 2006--2008 \\
    \emph{Program Manager}, Visual C++ Compiler
\end{itemize}

\section*{Products}

\paragraph{Related to Proposed Project}
\begin{enumerate}
  \item Peter-Michael Osera.
    \emph{Constraint-based Type-directed Program Synthesis}.
    In \emph{Type-directed Development (TyDe)}, 2019.
  \item Peter-Michael Osera.
    \emph{Programming Assistance for Type-directed Programming (Extended Abstract)}.
    In \emph{Type-driven Development (TyDe)}, 2016.
  \item Jonathan Frankle, Peter-Michael Osera, David Walker, and Steve Zdancewic.
    \emph{Example-Directed Synthesis: A Type-Theoretic Interpretation}.
    In \emph{Principles of Programming Languages (POPL)}, 2016.
  \item Peter-Michael Osera.
    \emph{Program Synthesis with Types}.
    PhD thesis, University of Pennsylvania, 2015.
  \item Peter-Michael Osera and Steve Zdancewic.
    \emph{Type-and-Example-Directed Program Synthesis}.
    In \emph{Programming Language Design and Implementation (PLDI)}, 2015.
\end{enumerate}

\paragraph{Selected Other Products}
\begin{enumerate}
  \item
    Brett A. Becker, Paul Denny, Raymond Pettit, Durell Bouchard, Dennis J. Bouvier, Brian Harrington, Amir Kamil, Amey Karkare, Chris McDonald, Peter-Michael Osera, Janice L. Pearce, and James Prather.
    \emph{Unexpected Tokens: A Review of Programming Error Messages and Design Guidelines for the Future}.
    In \emph{ACM Conference on Innovation and Technology in Computer Science Education (ITiCSE)}, 2019.
  \item
    David G. Wonnacott and Peter-Michael Osera.
    \emph{A Bridge Anchored on Both Sides: Formal Deduction in Introductory CS, and Code Proofs in Discrete Math.}
    CoRR abs/1907.04134 (arXiv).  2019.
  \item
    Barbara M. Anthony, Mia Minnes, David Liben-Nowell, and Peter-Michael Osera.
    \emph{Modernizing the Mathematics Taught in Computer Science}.
    In \emph{Symposium on Computer Science Education (SIGCSE)}, 2019.
  \item
    Peter-Michael Osera and David G.Wonnacott.
    \emph{A Blocks-based Language for Program Correctness Proofs}.
    In \emph{IEEE Blocks and Beyond Workshop (B\&B)}, 2017.
  \item
    Jianting Chen, Medha Gopalaswamy, Prabir Pradhan, Sooji Son, and Peter-Michael Osera.
    \emph{ORC$^2$A: A Proof Assistant for Undergraduate Education}.
    In \emph{Symposium on Computer Science Education (SIGCSE)}, 2017.
\end{enumerate}

\subsection*{Synergistic Activities}
\begin{enumerate}
  \item Panelist: \emph{Uncommon Teaching Languages} (SIGCSE 2016), \emph{Finding Your Kind of Teaching School: Different Paces at Different Places} (TAPIA 2016).
  \item Birds of a Feather Session Co-organizer: \emph{Mentoring Student Teaching Assistants for Computer Science} (SIGCSE 2016), \emph{Making Induction Meaningful, Recursively} (SIGCSE 2014).
  \item Reviewer: SIGCSE (Volunteer reviewer, 2014--2018; Demos and Lightning Talks Chair, 2018--2019; Associate Program Chair, 2020), Grace Hopper Celebration of Women in
    Computing---Scholarships (2015), PLDI (ERC, 2016), PLATEAU (PC, 2016), IFL (PC, 2018), TyDe (PC, 2018).
  \item Penn Center for Teaching and Learning Graduate Fellow for Teaching
    Excellence (2012--2013).
\end{enumerate}

\end{document}
