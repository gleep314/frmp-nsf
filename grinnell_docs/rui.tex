\documentclass[11pt]{article}

\newcommand{\thedoc}{RUI Statement}
\usepackage{proposal}

\begin{document}

\smalltitle{}

Grinnell College and its Department of Computer Science have a strong commitment to (1) preparing its graduates for careers in STEM, both academic and industry, and (2) fostering a diverse student population.
The proposed project provides the PI with the resources necessary to sustain a vibrant undergraduate programming languages research group at Grinnell and establish strong collaborative connections with neighboring instutitions, which will help strengthen Grinnell's commitment to career preparation and diversity.
\vspace{2ex}
\section{About Grinnell College}

Because of the College and the Department of Computer Science's commitment to graduate preparedness and diversity, both enjoy a distinguished track record of successful educational outcomes in STEM.
Over half of the graduating class of 2019 (54.8\%) had at least one STEM major.
Grinnell ranks 10th (by institutional yield ratio) among schools from which STEM PhDs received their bachelor's degrees, according to the Council of Independent Colleges.
Of these Grinnell alumni that went on to pursue doctorate degrees, 61 are NSF GRFP award-winners and 132 are honorable mentions.
Of students graduating with a CS degree from Grinnell between 2000 and 2014, 85\% are either working in a STEM field or pursuing a graduate degree in STEM.
Because of the College's commitment to reaching out to underrepresented groups, the incoming class of 2019 possesses 12.2\% students that identify as first-generation, 22.4\% students that identify as domestic students of color, and 19.8\% are international students.
The Department of Computer Science also takes great strides to reach out to underrepresented groups.
14.9\% of the current majors are domestic students of color, 47.9\% are international students, and notably 36\% are female, double the national average.

One of the main contributors to these excellent statistics is Grinnell College's commitment to innovative pedagogy, such as inquiry-based learning and workshop-style courses.
The Department of Computer Science is well known as a leader in computer science education.
Its successful and innovative curriculum~\cite{cowden:sigcse:2012, rebelsky:sigcse:2013} (partially supported by NSF CCLI-0633090) is recognized as an exemplar in the most recent revision of the ACM/IEEE Undergraduate Computer Science Curricular guidelines~\cite{acm:curriculum:2013}.
Because of these innovative approaches, the percentage of female computer science majors in the department (currently 44\%) is two to three times the national average.

Another contributor is the various programs the College hosts to engage students in the sciences.
For example, the Grinnell Science Project introduced several interventions to encourage incoming first-year students to pursue STEM careers, in particular, those students from underrepresented groups.
These interventions include a multi-day pre-orientation program for incoming first-year students who are interested in the sciences, substantial redesign of the introductory courses in the science (including computer science), peer mentoring, and other forms of support.
In 2011, the program was recognized with a Presidential Award for Excellence in Science, Mathematics, and Engineering Mentoring by the NSF.
The second-year science retreat is an opportunity for students interested in declaring a science-related major to sort out the logistics of declaring a major, develop a four-year plan, learn about research and study opportunities, and so forth.
The PI was a faculty volunteer in the past for this retreat.

Both the department and the College understand the need to actively engage students in research as part of their educational experience.
The Mentored Advanced Project (MAP) serves as the primary vehicle for students to get involved in science research at Grinnell.
In a MAP, a student proposes and conducts an independent research project under the guidance of a faculty member.
 The final results of the MAP are expected to be in a form suitable for presentation outside of the College.
Almost 40\% of Grinnell students and 50\% of science majors complete a MAP before they graduate.
Recognizing the need to attract students towards STEM early in their careers, Grinnell also offers directed research project opportunities to students completing their second year.

Computer science performs particularly well concerning MAPs.
With only 4.6 full-time employees, the department maintains approximately $11 \pm 2$ MAP students per summer over the last seven years.
Overall, 38\% of the computer science majors participate in a MAP.
In this time, these projects have resulted in over a dozen student co-authored publications.

\section{Impact}

As pre-tenure faculty at Grinnell College, the PI is establishing Grinnell College as a presence in undergraduate programming languages and systems research in the Midwest.
He co-leads the Systems and Languages @ Grinnell (SL@G) group alongside Charlie Curtsinger, where they investigate problems at the intersection of systems, programming languages, human-computing interaction, and computer science education.
In addition to cutting edge research, the group nurtures the next generation of computer scientists that specialize in programming languages and systems, whether they go on to industry to academia.
To do this, the PI expects to maintain his pipeline of summer student researchers, keeping programming languages research vibrant and alive at Grinnell.

\paragraph{Students}
The funding provided by this grant will help the PI maintain his target cohort of 4--6 students each summer.
(The PI will apply for competitively-awarded College funds to cover the remaining students every year.)
The PI believes in maintaining a research group of significant size every summer (SL@G hosts 8--12 students every summer between the PI and his co-leader) as there is substantial benefit in building a sizable research community of undergraduates.
With a significant group, student researchers can support each other by working together on a project, hashing out ideas informally in the lab, or merely commiserating over the lows and highs of research.
Also, with more number of students, the PI can justify larger-scale efforts to foster and maintain a research community among his undergraduate research group such as weekly seminars, internal mailing lists, bonding events, \etc.

Additional MAP opportunities also help fill a particular gap in the Grinnell computer science curriculum.
Like many other computer science departments in the nation, Grinnell is experiencing record enrollments.
However, the number of computer science faculty has not scaled up to meet this demand; as a result, it is challenging to offer upper-level courses in advanced programming language theory or compilers to give students deep expertise in the area.
MAPs provide an alternative venue for students to get that level of depth in their education.

The PI has a successful track record of working with undergraduate students.
Since 2016, he has hosted 24 undergraduate in his research group, 14 of which has moved on to top-tier positions in both industry and academia.
Three alumni of the PI's research group have gone on to pursue doctorate degrees in programming languages and related fields.
One student is currently applying to mathematics doctoral programs, and two students are planning to apply to doctorate programs in computer science post-graduation.
Students have presented their work both at regional conferences such as the Midwest Programming Languages Summit (MWPLS) and international venues for computer science education (SIGCSE) and programming languages (PLDI, ICFP).
Also, several students have attended the Programming Languages Mentoring Workshop (PLMW) held at various international programming languages conferences to learn more about the field and community.
These students have universally reported that these events were highly beneficial in obtaining post-graduate employment either in industry or academia.

\paragraph{The PI}
The project proposal is an excellent opportunity for the PI to evolve his research agenda.
The PI is deeply interested in taking his work in theoretical programming languages and program synthesis and finding useful applications of the technology these fields have developed.
The proposed project offers an excellent opportunity for the PI to develop a broader, interdisciplinary understanding of the relevance and potential of programming language theory to influence new and emerging technologies.
The project also offers excellent opportunities to collaborate with other research institutions in the Midwest, increasing the scope of the PI's work in the process.
Furthermore, the summer pay requested in the proposal will help the PI keep up their research productivity in light of the significant teaching load at Grinnell.

In addition to research, the project proposal travel budget will allow the PI to travel to national conferences specifically to continue nurturing collaborative relationships with the broader community.
Collaboration is essential for the PI (who is at a SLAC) to stay visible and relevant in his field.
This allows him to keep up to date with the latest trends as well as work more efficiently with research peers he would otherwise not have available at his institution.

\section{Plan of Action}

\paragraph{Recruiting}

To stay true to the department's mission of maintaining a diverse body of students, the PI does not just cherry-pick the top-performing students in his courses.
Thanks to the small, intimate learning environment fostered at Grinnell, he actively reaches out to students that he deems is capable and interested in programming languages research but are otherwise unsure of themselves or doubt their abilities.
In addition he also recruits a broad range of students not just in terms of experience---second-year and third-year students---but also students from underrepresented groups to help broaden the perspectives of not only his research group but the programming languages community as a whole.

\paragraph{Preparation for Summer Research}

Grinnell's first introductory course uses the Racket programming language, so all potential research students working with the PI are already familiar with functional programming.
Furthermore, Grinnell consistently offers coursework in programming language implementation as well as special topics in advanced programming languages concepts.
On top of these course offerings, the PI runs an informal programming languages seminar for his MAP students so that they can gain:
\begin{itemize}[itemsep=0pt]
  \item A basic knowledge of type theory and programming language implementation that serves as the foundation for all of the PI's research efforts.
  \item Fundamental skills in reading, organizing and analyzing academic writing, particularly the domain-specific prose of the PI's field.
  \item A sense of the expectations of research-level work and the soft skills (\eg, collaboration, wellness, and time management) necessary to carry out that work.
\end{itemize}
Also, the PI will hold ``ramp-up'' meetings with his students and collaborators at Drake and Iowa State University during the spring to introduce his students to the biological and chemical aspects of the research.

\paragraph{Assessment}
The PI is interested in assessing the learning outcomes of his research students and is actively investigating ways to do that in anticipation of the summer.
One such instrument the PI will likely adapt is the Survey of Undergraduate Research Experience (SURE)~\cite{lopatto:tcp:2008}~\footnote{%
  The SURE survey is no longer centrally administered as of 2018.
  However, the PI will discuss with the creator of the survey who is Grinnell faculty how to adapt the survey for his students.
} which will allow the PI to study the effects of the undergraduate research experience that he offers his students.

\section{Acknowledgments}
The PI would like to acknowledge and thank the college and department faculty and staff for providing the resources and statistics that were cited in this document.

\newpage

\bibliographystyle{abbrv}
\bibliography{refs}

\end{document}
