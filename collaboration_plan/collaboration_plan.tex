\documentclass[11pt]{article}

%!TEX root = ../project_description/project_description.tex
\usepackage[USenglish]{babel}

\usepackage[T1]{fontenc}      
\usepackage[utf8]{inputenc}

\usepackage{amsmath,amsthm}
\usepackage{amsfonts}
\usepackage{amssymb}
\usepackage{stmaryrd}
\usepackage{graphicx}
% \usepackage{titlesec}
\usepackage[margin=1in]{geometry}
\usepackage{subcaption}

\usepackage{xspace}
\newcommand{\ie}{i.e.\@\xspace}
\newcommand{\eg}{e.g.\@\xspace}

\makeatletter
\newcommand*{\etc}{%
    \@ifnextchar{.}%
        {etc}%
        {etc.\@\xspace}%
}
\makeatother

\makeatletter
\newcommand*{\etal}{%
    \@ifnextchar{.}%
        {et al}%
        {et al.\@\xspace}%
}
\makeatother

% \titleformat{\subsubsection}[runin]{\normalfont\large\bfseries}{\thesubsubsection}{1em}{}[]

\usepackage[textsize=scriptsize]{todonotes}

% \newlength{\arrow}
% \settowidth{\arrow}{\scriptsize$1000$}
\newcommand*{\goesto}[1]{\xrightarrow{#1}}
\newcommand*{\Xbar}{\overline{X}}
\newcommand*{\xbar}{\overline{x}}

\usepackage{hyperref}
\usepackage{listliketab}
\usepackage{array}
\usepackage{longtable}
\usepackage{enumitem}
\usepackage{sectsty}
\usepackage{url}
\usepackage{bibentry}
\usepackage{fancyhdr}
\usepackage{xparse}

\newenvironment{thoughts}{\begin{itemize}[itemsep=0pt]}{\end{itemize}}
\newcommand{\athought}[1]{\item \emph{#1}}

\RequirePackage[final]{microtype}
\DisableLigatures{encoding = T1, family = tt* }

\bibliographystyle{plain}


\begin{document}

    \setcounter{page}{1}
    \begin{center}
        {\Large {\bf Collaboration Plan}}
    \end{center}

      The PIs Klinge, Lathrop, and Osera are assistant professors at Drake University, Iowa State University, and Grinnell College, respectively.
      These institutions are all located within a 50 mile radius of each other.
      These three individuals will assume overall supervision of the proposed research.
      PI Lathrop and PI Klinge have relevant expertise in the theory of chemical reaction networks, PI Lathrop has some expertise in the software engineering of chemical reaction networks, and also industry experience in the engineering of large systems.
      PI Osera has relevant expertise in the theory of programming languages.
    
      Throughout the course of the proposed research PI Klinge will supervise 3 undergraduate students and PI Osera will supervise 2 undergraduate students each year at their respective institutions.
      PI~Lathrop and PI Klinge will co-supervise Ph.D. students from Iowa State University to help with the proposed research and mentoring of undergraduate students.
      In addition, two undergraduate research assistants from Iowa State University will participate in the proposed research during summers that the project is active.
      
      Three levels of communication will help facilitate the coordination of the PIs and students between the three institutions during the course of this proposed research.
      \begin{enumerate}
        \item
        Monthly on-site meetings that include formal presentations on the progress of the project.
        These meetings will last most of a day and provide valuable face-to-face time for the participants in the proposed research to address any difficult or critical issues and present new ideas and results to others.
        These monthly meetings will also help foster a sense of community among the diverse participants from the three institutions.
        The budget from each institution includes the necessary funds to travel to the other institutions.

        \item
        Weekly videoconference meetings between all participants that include short progress reports from each member.
        These meetings will give each participant prompt feedback on their progress from the other members of the project.
        This is especially important because of the interdisciplinary nature of the project and to keep the project moving forward.

        \item
        Real time persistent communication during the week via collaborative messaging tools such as Microsoft Teams, Slack, and collaborative code editing tools such as Microsoft Visual Studio Code's LiveShare extension.
        Tools such as Microsoft Teams and Slack preserve all communications that can easily be searched and archived.
        It is anticipated that this informal communication medium will enhance the frequency and duration of interactions between collaborators at different institutions.
      \end{enumerate}
      
      The graduate research assistants will maintain elevated communication levels with undergraduate students at the three institutions.
      This will help keep undergraduate students working on the proposed research focused and any questions or issues addressed quickly.
      This also gives the graduate assistants experience in mentoring students under the supervision of PI Lathrop and PI Klinge.
       
      The proposed research will utilize software servers and mirrored repositories at each of the institutions to share collaborative software, research artifacts, and publications.
      Extensive use of version control software such as git will be employed, making it seamless for the three institutions to collaborate.

      During each summer of the project, undergraduate student participants will be employed full-time.
      Funds are allocated for each PI for these summers to help supervise and direct the research during these time-intensive months.
    
\end{document}